% Requested

Este relatório detalha o projeto desenvolvido para o segundo trabalho prático da unidade curricular de Programação em Lógica do MIEIC-FEUP, na qual se explora o puzzle Gold Star e como este pode ser resolvido com auxilio a programação em lógica por restrições. O objetivo inicial proposto para este trabalho foi de criar um ficheiro que contenha todas as soluções possíveis do puzzle Gold Star, partindo de uma estrela de cinco pontas. Não sabendo o grau de exigência computacional que este problema poderia ou não ter, tornou-se um desafio interessante da qual conseguimos retirar conclusões satisfatórias.

O relatório começa por descrever o problema estudado, passando para a abordagem tomada por nós para o resolver, contendo detalhes sobre as variáveis de decisão usadas e as restrições aplicadas. Após isto é brevemente mencionado como é feita a representação dos problemas e soluções, acabando com uma análise das experiências executadas, junto com as conclusões possíveis de retirar dos dados obtidos.