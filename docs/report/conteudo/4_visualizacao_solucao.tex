% Requested

Para visualizar uma estrela de cinco pontas é possível usar o predicado \verb|print_star/2| para imprimir a configuração e os resultados da estrela na consola do SICStus (Fig. \ref{fig: representacoes}).

Este desenho foi codificado à força para ficar apelativo ao representar uma estrela de cinco pontas. Para estrelas com um número diferente de pontas seria necessário fazer uma função de visualização para cada. Como esse não é o objectivo deste projecto, estes predicados não foram elaborados. 

Invés disso, para representar a configuração e a solução dessa configuração, as duas listas são impressas na consola, separadas por um espaço, na mesma linha, usando o predicado \verb|print_result/2|. Esta forma é compacta e simples, componentes necessárias quando for necessário guardar várias destas listas num ficheiro.