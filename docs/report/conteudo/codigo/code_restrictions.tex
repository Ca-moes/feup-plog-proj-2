\begin{figure}
    \centering
    \begin{tabular}{c}
    \begin{lstlisting}[language=Prolog]
    apply_restriction(Op0i, C, A, Op7i, I, F),
    apply_restriction(Op1i, A, D, Op4i, G, J),
    apply_restriction(Op9i, B, C, Op2i, D, E),
    apply_restriction(Op8i, B, F, Op5i, H, J),
    apply_restriction(Op3i, G, E, Op6i, I, H).
    
    % predicate to apply the equation restriction
    apply_restriction(Op1, Var1, Var2, Op2, Var3, Var4):-
        apply_restriction(Op1, Var1, Var2, Value),
        apply_restriction(Op2, Var3, Var4, Value).
    apply_restriction(+, Var1, Var2, Value):-
        Var1+Var2 #= Value.
    apply_restriction(-, Var1, Var2, Value):-
        Var1-Var2 #= Value.
    apply_restriction(*, Var1, Var2, Value):-
        Var1*Var2 #= Value.
    % in case of division, the operands must be integers
    apply_restriction(/, Var1, Var2, Value):-
        % Var1/Var2 #= Value
        Var1 #= Value*Var2.
    \end{lstlisting}
    \end{tabular}
    \caption{Nas primeiras 5 linhas está uma versão desactualizada de como se estavam a definir as restrições dos operandos, usando o predicado apply\_restriction/6 fica simples aplicar as restrições a cada equação}
    \label{code:predicados_restricoes}
\end{figure}