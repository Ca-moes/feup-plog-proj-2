% Requested

Este projeto serviu para demonstrar que um problema que parece simples numa primeira vista consegue conter muita informação possível de analisar, que foi demonstrado pela análise dimensional e pela análise de estratégias de pesquisa.

Gostaríamos de realçar o método exemplificado na secção 3.2, que não foi fácil de descobrir e que possibilitou ignorar o conceito de equações em formato de estrela para poder formar os predicados que aplicam as restrições aos operandos, apenas baseando-se nas posições dos operadores e operandos nas suas listas respectivas.

Por fim, deixamos como trabalho futuro uma restrição que não conseguimos criar em prolog durante o desenvolvimento deste projeto: 

Com a restrição \verb|bigger| são calculadas configurações a mais que equivalem à mesma configuração com um deslocamento em todos os operadores, por exemplo:

\begin{figure}
    \centering
    \begin{tabular}{c}
    \begin{lstlisting}[language=Prolog]
    [-,+,-,-,-,-][2,3,4,5,0,1]
    [-,-,+,-,-,-][1,2,3,4,5,0]
    [-,-,-,+,-,-][0,1,2,3,4,5]
    [-,-,-,-,+,-][5,0,1,2,3,4]
    [-,-,-,-,-,+][4,5,0,1,2,3] 
    \end{lstlisting}
    \end{tabular}
    \label{code:future_work}
\end{figure}

Apenas uma destas configurações seria necessária de calcular, mas não conseguimos achar um método que filtre correctamente as configurações de acordo com o pretendido.